%%%%%%%%%%%%%%%%%%%%%%%%%%%%%%%%%%%%%%%%%
% Stylish Article
% LaTeX Template
% Version 2.1 (1/10/15)
%
% This template has been downloaded from:
% http://www.LaTeXTemplates.com
%
% Original author:
% Mathias Legrand (legrand.mathias@gmail.com) 
% With extensive modifications by:
% Vel (vel@latextemplates.com)
%
% License:
% CC BY-NC-SA 3.0 (http://creativecommons.org/licenses/by-nc-sa/3.0/)
% Adaptation: Jens Buysse
%
%%%%%%%%%%%%%%%%%%%%%%%%%%%%%%%%%%%%%%%%%


\documentclass[fleqn,10pt]{voorstel}

\usepackage[english,dutch]{babel}

\usepackage{lipsum}

\setlength{\columnsep}{0.55cm} % Distance between the two columns of text
\setlength{\fboxrule}{0.75pt} % Width of the border around the abstract

%----------------------------------------------------------------------------------------
%	COLORS
%----------------------------------------------------------------------------------------

\definecolor{color1}{RGB}{0,111,184} % Color of the article title and sections
\definecolor{color2}{RGB}{0,20,20} % Color of the boxes behind the abstract and headings

\usepackage{hyperref} % Required for hyperlinks
\hypersetup{hidelinks,colorlinks,breaklinks=true,urlcolor=color1,citecolor=color2,linkcolor=color2,bookmarksopen=false,pdftitle={Title},pdfauthor={Author}}

%----------------------------------------------------------------------------------------
%	ARTICLE INFORMATION
%----------------------------------------------------------------------------------------

\JournalInfo{Hogeschool Gent} % Journal information
\Archive{Onderzoekstechnieken 2016 - 2017} % Additional notes (e.g. copyright, DOI, review/research article)

\PaperTitle{Verslag brainstormsessies} % Article title

% Authors
\Authors{
	Brian Pinsard, 
	Jovi De Croock, 
	Thomas Vansevenant, 
	Dieter Willems
	}
	
% Author affiliation ()
%\affiliation{\textsuperscript{1}\textit{Uitleg bij * of superscript}}

% Keywords 
\Keywords{
	Mobiele applicatieontwikkeling --- 
	Webapplicatieontwikkeling
	} 
\newcommand{\keywordname}{Onderzoeksdomeinen} % Defines the keywords heading name


\Abstract{Hier schrijf je de samenvatting van uw voorstel. Wat hier zeker in moet vermeld worden:
\begin{enumerate}
	\item \textbf{Context}:  Waarom is dit werk belangrijk?
	\item \textbf{Nood} :  Waarom moet dit onderzocht worden?
	\item \textbf{Taak}:  Wat ga je (ongeveer) doen?
	\item \textbf{Object}: Wat staat in dit document geschreven?
	\item \textbf{Resultaat}: Wat verwacht je van je onderzoek?
	\item \textbf{Conclusie}: Wat verwacht je van van de conclusies?
	\item \textbf{Perspectief}: Wat zegt de toekomst voor dit werk?
\end{enumerate}

Bij de keywoorden beschrijf je het domein, samen met andere keywords die je werk beschrijven.
}

%----------------------------------------------------------------------------------------

\begin{document}

\flushbottom % Makes all text pages the same height

\maketitle % Print the title and abstract box

\tableofcontents % Print the contents section

\thispagestyle{empty} % Removes page numbering from the first page

%----------------------------------------------------------------------------------------
%	ARTICLE CONTENTS
%----------------------------------------------------------------------------------------

\section{Introductie} % The \section*{} command stops section numbering
Wij werken met drie onder de cluster mobiele applicatieontwikkeling en één van ons werkt onder de cluster webapplicatieontwikkeling. Brian, Thomas en Dieter hebben de opleidingsonderdelen mobiele applicatieontwikkeling en webapplicatieontwikkeling
al gevolgd waardoor ze hun interessegebied reeds hebben kunnen afbakenen, Jovi is van plan dit het komende semester op te nemen en heeft aan de hand van informatie van medestudenten zijn besluit kunnen nemen.
Diegene die mobiele applicatieontwikkeling hebben gekozen is met de reden dat dit dichter bij hun interessegebied ligt, Thomas heeft om dezelfde reden gekozen voor webapplicatieontwikkeling.

\section{Strategiebeschrijving}
Bespreek en reflecteer over de strategie die jullie gebruikt hebben voor het zoeken naar informatie.

\section{Retrospectief}
\subsection{Wat ging goed?}
We hebben verschillende communicatie platformen gebruikt om elkaar te bereiken. Dit verliep zeer goed en hierdoor hebben we een tweede brainstorm sessie kunnen organiseren.
We hebben relatief makkelijk onze interesse gebieden kunnen afbakenen en beslissen welke richting we uit wilden.
\subsection{Welke bronnen bleken interessant?}
...
\subsection{Welke bronnen waren het meest inspirerend?}
...
\subsection{Tips om te onthouden?}
...
\subsection{Tegenvallers?}
Bij de eerste brainstormsessie waren Thomas en Dieter afwezig omwille van omstandigheden, dit heeft geleidtot het organiseren van een nieuwe brainstormsessie met alle vier de groepsleden aanwezig. In de groep waren er ook
moeilijkeden met beslissen waardoor de sessie wat langer duurde dan het zou horen.
\subsection{valkuilen?}
...
\subsection{Zijn er zaken waar jullie nu nog mee worstelen?}
...

\section{Conclusie}
...

%----------------------------------------------------------------------------------------
%	REFERENCE LIST
%----------------------------------------------------------------------------------------
\phantomsection
\bibliographystyle{apa}
\bibliography{biblio}

%----------------------------------------------------------------------------------------

\end{document}
