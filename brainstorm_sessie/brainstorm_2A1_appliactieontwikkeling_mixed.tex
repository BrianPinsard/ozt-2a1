%%%%%%%%%%%%%%%%%%%%%%%%%%%%%%%%%%%%%%%%%
% Stylish Article
% LaTeX Template
% Version 2.1 (1/10/15)
%
% This template has been downloaded from:
% http://www.LaTeXTemplates.com
%
% Original author:
% Mathias Legrand (legrand.mathias@gmail.com) 
% With extensive modifications by:
% Vel (vel@latextemplates.com)
%
% License:
% CC BY-NC-SA 3.0 (http://creativecommons.org/licenses/by-nc-sa/3.0/)
% Adaptation: Jens Buysse
%
%%%%%%%%%%%%%%%%%%%%%%%%%%%%%%%%%%%%%%%%%


\documentclass[fleqn,10pt]{voorstel}

\usepackage[english,dutch]{babel}

\usepackage{lipsum}

\setlength{\columnsep}{0.55cm} % Distance between the two columns of text
\setlength{\fboxrule}{0.75pt} % Width of the border around the abstract

%----------------------------------------------------------------------------------------
%	COLORS
%----------------------------------------------------------------------------------------

\definecolor{color1}{RGB}{0,111,184} % Color of the article title and sections
\definecolor{color2}{RGB}{0,20,20} % Color of the boxes behind the abstract and headings

\usepackage{hyperref} % Required for hyperlinks
\hypersetup{hidelinks,colorlinks,breaklinks=true,urlcolor=color1,citecolor=color2,linkcolor=color2,bookmarksopen=false,pdftitle={Title},pdfauthor={Author}}

%----------------------------------------------------------------------------------------
%	ARTICLE INFORMATION
%----------------------------------------------------------------------------------------

\JournalInfo{Hogeschool Gent} % Journal information
\Archive{Onderzoekstechnieken 2016 - 2017} % Additional notes (e.g. copyright, DOI, review/research article)

\PaperTitle{Verslag brainstormsessies} % Article title

% Authors
\Authors{
	Brian Pinsard, 
	Jovi De Croock, 
	Thomas Vansevenant, 
	Dieter Willems
	}
	
% Author affiliation ()
%\affiliation{\textsuperscript{1}\textit{Uitleg bij * of superscript}}

% Keywords 
\Keywords{
	Mobiele applicatieontwikkeling --- 
	Webapplicatieontwikkeling
	} 
\newcommand{\keywordname}{Onderzoeksdomeinen} % Defines the keywords heading name


\Abstract{
	Als student bachelor toegepaste informatica wordt er van ons verwacht dat we een bachelorproef schrijven, vooraleer de student mag starten met zijn bachelorproef op te stellen moet deze een bachelorproef voorstel opstellen en moet deze voldoen aan de aanvaardingscriteria. Dit werk zorgt ervoor dat ieder binnen onze groep met behulp van brainstormsessies en feedback van medestudenten tot een eigen gewenst onderwerp komt en zo zijn bachelorproef voorstel kan opstellen. De eerste brainstormsessie zoals in dit verslag beschreven staat geeft ons een algemene richting waar ons onderwerp van het bachelorproef voorstel uit zal volgen. De tweede sessie bestaat uit het toelichten van de voorstellen aan elkaar zodat er constructieve feedback kan volgen. Na beide sessies is iedereen klaar om zijn eigen bachelorproef voorstel op te stellen. Dit voorstel kan dan dienen als startpunt voor het finale bachelorproef voorstel of als richtlijn indien de student toch een andere richting in wil gaan voor het echte werk. 
	Hier schrijf je de samenvatting van uw voorstel. Wat hier zeker in moet vermeld worden:
}

%----------------------------------------------------------------------------------------

\begin{document}

\flushbottom % Makes all text pages the same height

\maketitle % Print the title and abstract box

\tableofcontents % Print the contents section

\thispagestyle{empty} % Removes page numbering from the first page

%----------------------------------------------------------------------------------------
%	ARTICLE CONTENTS
%----------------------------------------------------------------------------------------

\section{Introductie} % The \section*{} command stops section numbering
Wij werken met drie onder de cluster mobiele applicatieontwikkeling en één van ons werkt onder de cluster webapplicatieontwikkeling. Brian, Thomas en Dieter hebben de opleidingsonderdelen mobiele applicatieontwikkeling en webapplicatieontwikkeling
al gevolgd waardoor ze hun interessegebied hebben kunnen bepalen, Jovi is van plan dit het komende semester op te nemen en heeft aan de hand van informatie van medestudenten zijn besluit kunnen nemen.
Diegene die mobiele applicatieontwikkeling hebben gekozen is met de reden dat dit dichter bij hun interessegebied ligt, Thomas heeft om dezelfde reden gekozen voor webapplicatieontwikkeling.

\section{Strategiebeschrijving}
We zijn gestart met het afbakenen van onze interesse domeinen, bij de eerste brainstormsessie was dit nog steeds zeer uitgebreid maar dit hebben we kunnen verkleinen zodat ieder persoon binnen de groep een eigen interessedomein had. \\
Het afbakenen van onze interesse domeinen resulteerde in het volgende: Brian zal zich verdiepen in multiplayer game development voor het android platform, Jovi in het analyseren van de performantie van android applicaties, Thomas in internet of things gerelateerd met smart cities en Dieter in smartphones als bedieningsapparaten. \\
Nadat ieder zijn interesse domein bepaald had, hebben we gestart met op zoek te gaan naar standaardwerken. Hiervoor hebben we gebruik gemaakt van de verschillende bronnen die beschikbaar gesteldt worden door het Apollo portaal, voornamelijk de scripties die beschikbaar zijn van oud studenten op  de databank van HoGent en de zoekmachine Google Scholar waarin zeer handige bronnen. \\
Bij dit opzoekwerk heeft ieder zijn interessantste bronnen bijgehouden als referentie door gebruik te maken van een bibtex bestand, welke we nadien samengevoegd hebben tot één algemeen bibtex bestand.

\section{Retrospectief}
\subsection{Wat ging goed?}
We hebben verschillende communicatie platformen gebruikt om elkaar te bereiken. Dit verliep zeer goed en hierdoor hebben we een tweede brainstorm sessie kunnen organiseren.
We hebben relatief makkelijk onze interesse gebieden kunnen afbakenen en beslissen welke richting we uit wilden.
\subsection{Welke bronnen bleken interessant?}
...
\subsection{Welke bronnen waren het meest inspirerend?}
...
\subsection{Tips om te onthouden?}
Het belangrijkste lijkt ons dat enkel secundaire bronnen bruikbaar zijn als referenties en dat men deze referenties makkelijk kan beheren door middel van een tool zoals jabref of mendeley.
\subsection{Tegenvallers?}
Bij de eerste brainstormsessie waren Thomas en Dieter afwezig omwille van omstandigheden, dit heeft geleidtot het organiseren van een nieuwe brainstormsessie met alle vier de groepsleden aanwezig. In de groep waren er ook
moeilijkeden met beslissen waardoor de sessie wat langer duurde dan het zou horen.
\subsection{valkuilen?}
Tijdens onze brainstormsessie hebben we ondervonden dat we tijdens het afbakenen van het interesse domein vaak tot een toestand van besluiteloosheid komen, dit hebben we dan steeds opgelost door pro's en contra's van elk domein te overwegen en zo een besluit te nemen. 
\subsection{Zijn er zaken waar jullie nu nog mee worstelen?}
...

\section{Conclusie}
Iedereen binnen de groep heeft vrij vlot zijn interesse domein kunnen afbakenen en heeft zo een duidelijke richting waarin hij zal voortwerken, hieruit volgend zal ieder zijn eigen onderwerp bepalen en een korte presentatie hier rond opstellen. De volgende sessie zal ieder zijn voorstel toelichten aan de hand van deze presentatie en zal er contructieve feedback volgen.

%----------------------------------------------------------------------------------------
%	REFERENCE LIST
%----------------------------------------------------------------------------------------
\phantomsection
\bibliographystyle{apa}
\bibliography{biblio}

%----------------------------------------------------------------------------------------

\end{document}
