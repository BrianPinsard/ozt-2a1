%%%%%%%%%%%%%%%%%%%%%%%%%%%%%%%%%%%%%%%%%
% Stylish Article
% LaTeX Template
% Version 2.1 (1/10/15)
%
% This template has been downloaded from:
% http://www.LaTeXTemplates.com
%
% Original author:
% Mathias Legrand (legrand.mathias@gmail.com) 
% With extensive modifications by:
% Vel (vel@latextemplates.com)
%
% License:
% CC BY-NC-SA 3.0 (http://creativecommons.org/licenses/by-nc-sa/3.0/)
% Adaptation: Jens Buysse
%
%%%%%%%%%%%%%%%%%%%%%%%%%%%%%%%%%%%%%%%%%


\documentclass[fleqn,10pt]{voorstel}

\usepackage[english,dutch]{babel}

\usepackage{lipsum}
\setlength{\columnsep}{0.55cm} % Distance between the two columns of text
\setlength{\fboxrule}{0.75pt} % Width of the border around the abstract

%----------------------------------------------------------------------------------------
%	COLORS
%----------------------------------------------------------------------------------------

\definecolor{color1}{RGB}{0,111,184} % Color of the article title and sections
\definecolor{color2}{RGB}{0,20,20} % Color of the boxes behind the abstract and headings

\usepackage{hyperref} % Required for hyperlinks
\hypersetup{hidelinks,colorlinks,breaklinks=true,urlcolor=color1,citecolor=color2,linkcolor=color2,bookmarksopen=false,pdftitle={Title},pdfauthor={Author}}

%----------------------------------------------------------------------------------------
%	ARTICLE INFORMATION
%----------------------------------------------------------------------------------------

\JournalInfo{Hogeschool Gent} % Journal information
\Archive{Onderzoekstechnieken 2016 - 2017} % Additional notes (e.g. copyright, DOI, review/research article)

\PaperTitle{Verslag presentaties} % Article title

% Authors
\Authors{
	Brian Pinsard, 
	Jovi De Croock, 
	Thomas Vansevenant, 
	Dieter Willems
	}
	
% Author affiliation ()
%\affiliation{\textsuperscript{1}\textit{Uitleg bij * of superscript}}

% Keywords 
\Keywords{
	Mobiele applicatieontwikkeling --- 
	Web applicatieontwikkeling
	} 
\newcommand{\keywordname}{Onderzoeksdomeinen} % Defines the keywords heading name


\Abstract{
	Samenvatting hier...
}

%----------------------------------------------------------------------------------------

\begin{document}

\flushbottom % Makes all text pages the same height

\maketitle % Print the title and abstract box

\tableofcontents % Print the contents section

\thispagestyle{empty} % Removes page numbering from the first page

%----------------------------------------------------------------------------------------
%	ARTICLE CONTENTS
%----------------------------------------------------------------------------------------

\section{Introductie} % The \section*{} command stops section numbering
Introductie hier...
Presentaties volgens ignite-formaat, ignite formaat is x y z zorgt ervoor dat z y x etc etc

\section{Strategiebeschrijving}
Strategiebeschrijving hier...
Planning (presentaties soft-deadline opgesteld, datum afgesproken, presentaties geven, feedback aan elkaar, verslag opstellen met de verworven feedback, etc...)

\section{Retrospectief}
\subsection{Wat ging goed?}
Soft-deadline zorgde voor bla bla en samenwerking was terug zeer vlot etc etc

\subsection{Feedback}
Zoals eerder vermeld hebben alle groepsleden na elke presentatie feedback gegeven hierbij ieder zijn retrospectieve over deze feedback... etc etc

%%Thomas
\paragraph{Thomas}
Tekst van Thomas hier...

%%Dieter
\paragraph{Dieter}
Tekst van Dieter hier...

%%Brian
\paragraph{Brian}
Tekst van Brian hier...

%%Jovi
\paragraph{Jovi}
Tekst van Jovi hier...

\subsection{Tips om te onthouden?}
Voor ignite formaat toe te passen kan je met behulp van een creative techniek zoals mindmapping eerst en vooral 
duidelijk keywords formuleren dit maakt het makkelijke etc etc

\subsection{Tegenvallers?}
Geen idee... -.-

\subsection{Valkuilen?}
Ignite formaat moeilijker dan verwacht, 20 slides etc etc

\subsection{Zijn er zaken waar jullie nu nog mee worstelen?}
Correctheid van onze voorstellen hun titels en onderwerpen, aanvaardbaar etc etc

\section{Conclusie}
Conclusie hier....
%----------------------------------------------------------------------------------------
%	REFERENCE LIST
%----------------------------------------------------------------------------------------
\phantomsection
%----------------------------------------------------------------------------------------

\end{document}
