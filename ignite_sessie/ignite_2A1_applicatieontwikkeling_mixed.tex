%%%%%%%%%%%%%%%%%%%%%%%%%%%%%%%%%%%%%%%%%
% Stylish Article
% LaTeX Template
% Version 2.1 (1/10/15)
%
% This template has been downloaded from:
% http://www.LaTeXTemplates.com
%
% Original author:
% Mathias Legrand (legrand.mathias@gmail.com) 
% With extensive modifications by:
% Vel (vel@latextemplates.com)
%
% License:
% CC BY-NC-SA 3.0 (http://creativecommons.org/licenses/by-nc-sa/3.0/)
% Adaptation: Jens Buysse
%
%%%%%%%%%%%%%%%%%%%%%%%%%%%%%%%%%%%%%%%%%


\documentclass[fleqn,10pt]{voorstel}

\usepackage[english,dutch]{babel}

\usepackage{lipsum}
\usepackage{natbib}  

\setlength{\columnsep}{0.55cm} % Distance between the two columns of text
\setlength{\fboxrule}{0.75pt} % Width of the border around the abstract

%----------------------------------------------------------------------------------------
%	COLORS
%----------------------------------------------------------------------------------------

\definecolor{color1}{RGB}{0,111,184} % Color of the article title and sections
\definecolor{color2}{RGB}{0,20,20} % Color of the boxes behind the abstract and headings

\usepackage{hyperref} % Required for hyperlinks
\hypersetup{hidelinks,colorlinks,breaklinks=true,urlcolor=color1,citecolor=color2,linkcolor=color2,bookmarksopen=false,pdftitle={Title},pdfauthor={Author}}

%----------------------------------------------------------------------------------------
%	ARTICLE INFORMATION
%----------------------------------------------------------------------------------------

\JournalInfo{Hogeschool Gent} % Journal information
\Archive{Onderzoekstechnieken 2016 - 2017} % Additional notes (e.g. copyright, DOI, review/research article)

\PaperTitle{Verslag presentaties} % Article title

% Authors
\Authors{
	Brian Pinsard, 
	Jovi De Croock, 
	Thomas Vansevenant, 
	Dieter Willems
	}
	
% Author affiliation ()
%\affiliation{\textsuperscript{1}\textit{Uitleg bij * of superscript}}

% Keywords 
\Keywords{
	Mobiele applicatieontwikkeling --- 
	Web applicatieontwikkeling
	} 
\newcommand{\keywordname}{Onderzoeksdomeinen} % Defines the keywords heading name


\Abstract{
	Ieder van ons heeft zijn onderwerp goed toegelicht en heeft een min of meer duidelijk beeld op wat hij wil onderzoeken voor zijn bachelorproef en wat hij hier mee zal bereiken. Zo doet Brian het over een Realtime Multiplayer Android game, Dieter over het mobiel bedienen van Auto’s, Thomas over Smart Garbage Collection in Smart cities en Jovi over performantie analyse aan de hand van third party tools in java en android. De feedback staat bij ieder omschreven en zal dienen als steun bij het maken van het individuele bachelorproef voorstel.
}

%----------------------------------------------------------------------------------------

\begin{document}

\flushbottom % Makes all text pages the same height

\maketitle % Print the title and abstract box

\tableofcontents % Print the contents section

\thispagestyle{empty} % Removes page numbering from the first page

%----------------------------------------------------------------------------------------
%	ARTICLE CONTENTS
%----------------------------------------------------------------------------------------

\section{Introductie} % The \section*{} command stops section numbering
Naar aanleiding van onze eerste brainstorm waarbij ieder een algemene richting kunnen vormen heeft rond zijn bachelorproef onderwerp hebben we een sessie georganiseerd met alle groepsleden waarbij we elkaars presentatie beoordeeld hebben, de feedback kan ieder groepslid gebruiken om zijn voorstel voor zijn bachelorproef onderwerp correcter op te bouwen.\\
De presentaties zijn gemaakt volgens het Ignite-formaat. Een Ignite presentatie bestaat uit 20 slides met 15 seconden tijd per slide. Het resultaat is dat elke presentatie maar 5 minuten duurt, dit zorgt ervoor dat de spreker het publiek letterlijk kan warm maken voor zijn idee.

\section{Strategiebeschrijving}
We hebben besloten om deze sessie na de paasvakantie in te plannen. Iedereen moest dus tegen het einde van de paasvakantie klaar zijn met zijn presentatie en een richtlijn hebben voor het onderwerp van zijn bachelorproef voorstel. De reden om het na de paasvakantie te doen was zodat iedereen voldoende tijd had voor research omtrent zijn domein om zo betere presentaties te verkrijgen.\\
Zo gezegd zo gedaan iedereen had tegen het einde van de paasvakantie zijn presentatie en we hebben dan besloten om op 4 April, 2016 een lokaal te huren voor onze presentaties. Dit ging zeer vlot en er is heel wat nuttige feedback gegeven waarmee iedereen dan aan de slag kon gaan.

\section{Retrospectief}
\subsection{Wat ging goed?}
Soft-deadline zorgde voor bla bla en samenwerking was terug zeer vlot etc etc

\subsection{Feedback}

%%Thomas
\paragraph{\cite{VanSevenant2016}}
De onderzoeksvraag van Thomas is heel specifiek en geeft duidelijk weer over wat zijn bachelorproef zal gaan. Thomas heeft een een goed idee over hoe hij zijn proof of concept zal uitwerken. Hij weet heel goed welke technieken, apparatuur en tools hij hiervoor zal gebruiken. Een minpunt in zijn onderzoeksvraag is machine learning. Dit kan hij beter weglaten, want dit is echter een vakgebied an sich en hierdoor zou de bachelorproef niet concreet genoeg zijn en te ver uitweiden. Zijn presentatie werd op een verstaanbare en overzichtelijke manier gegeven.

%%Dieter
\paragraph{\cite{Willems2016}}
Hij heeft een idee waar hij naartoe wilt maar is nog niet specifiek genoeg in welk deel van zijn domein hij hem nu werkelijk wil verdiepen. Hierbij hebben we hem aangeraden om wat dieper in één van zijn goed uitgelegde en beargumenteerde stukken te gaan. Doordat hij nog niet specifiek genoeg was, was zijn onderzoeksvraag ook zeer onduidelijk.\\
Wat dan wel weer zeer goed was uitgewerkt was zijn state of the art. Hij had zeer veel goede voorbeelden van wat een smartphone op dit moment allemaal al kan doen met een auto.

%%Brian
\paragraph{\cite{Pinsard2016}}
Weet al duidelijk welke richting hij uit wil met zijn bachelorproef, de presentatie was ook zeer duidelijk en goed opgebouwd. Tijdsaanduiding aan de hand van symbolen onderaan bij iedere slide zorgde ervoor dat het Ignite-formaat duidelijk gevolgd werd.\\
Bij het onderwerp voor zijn bachelorproef “Case study: Multiplayer game development met Google Play Services in Unity3D voor het Android platform” ontbreekt juist zijn keuze voor een real-time multiplayer type spel zoals wel aangetoond werd in zijn slides.

%%Jovi
\paragraph{\cite{DeCroock2016}}
Hij had een goed idee waar hij naartoe wou, maar zijn onderzoeksvraag was nog niet duidelijk geformuleerd. 
Zijn idee bestaat uit het testen van de performantie tussen Java en Android code, hoe snel eenzelfde stuk code met hetzelfde doel in Java en Android verwerkt word, dit zou gedaan worden via third party tools niet met de performance analytics van het IDE zelf. De voorbeelden van code die getest zouden worden zijn: basis code zoals bijvoorbeeld iteraties, recursies, \dots en databank vergelijking.

\subsection{Tips om te onthouden?}
Voor Ignite formaat toe te passen kan je met behulp van een creative techniek zoals mindmapping eerst en vooral duidelijk keywords formuleren dit maakt het opstellen van de presentatie een stuk makkelijker.

\subsection{Valkuilen?}
Bij sommige groepsleden kon de onderzoeksvraag net iets specifieker. De onderzoeksvraag mocht dieper ingaan op het onderwerp. Door in dialoog met elkaar te treden en opbouwende kritiek te geven, kregen zij een beter idee welke richting ze precies wilden inslaan.Twintig slides volgens het Ignite formaat uitwerken met daarbij enkel of zoveel mogelijk gebruik maken van afbeeldingen om ideeën te presenteren was minder evident dan verwacht. De meesten van de groep kwamen niet aan het aantal gewenste slides.

\section{Conclusie}
In het algemeen ging alles zeer vlot mede dankzij de zelf opgelegde deadline voor de presentaties, iedereen heeft opbouwende kritiek verkregen over zijn presentatie en kan deze gebruiken om zijn bachelorproef voorstel op te bouwen. 
%----------------------------------------------------------------------------------------
%	REFERENCE LIST
%----------------------------------------------------------------------------------------
\phantomsection
\bibliographystyle{apa}
\bibliography{presentaties_biblio}
%----------------------------------------------------------------------------------------


\end{document}
