Voor er aan dit onderzoek kon begonnen worden moesten we eerst weten wat een benchmark nu eigenlijk inhoud. In het algemeen is benchmarking een middel om kwaliteitsvolle en kostenefficiëntie van een product, methode of een proces te achterhalen. Het kan ook dienen als een managment tool om over tijd continue verbeteringen aan te brengen \citep{PIBenchmarking}. Bij ons is het een middel om na te gaan welke laptop de beste 3D capaciteiten heeft binnen onze groep.

\subsection{Vergelijkende studies}
\subsubsection{Characterization and Performance Analysis for 3D Benchmarks }
In 2012 is er een gelijkaardige studie gebeurd door Joseph Issa \citep{CPUGPU12}. Zij concluderen dat de geschatte prestatie fouten voor alle testcases minder is dan 5\% ten opzichte van gemeten prestatie data. Ze hebben ook uitgezocht wat de maximale performance van een proces op een gegeven benchmark. Bij ons onderzoek gaan we meer opzoek welke laptop er de beste is op vlak van 3D capaciteiten. Ons onderzoek kan zeer handig zijn om na te gaan welke laptops er binnen de richting toegepaste informatice worden gekocht en na te gaan welke er misschien het beste wordt gekocht moesten er studenten een nieuwe laptop nodig hebben.

\subsubsection{3D Graphics Benchmarks for Low-Power Architectures}
In 2002 is er een onderzoek naar 3D Graphics Benchmarks for Low-Power Architectures door Iosif Antochi, Ben Juurlink, Stamatis Vassiliadis and Petri Liuha \citep{lowPower}. Deze onderzoek is een goede representatie van wat een operationeel onderzoek zou kunnen zijn zowel inhoud als structuur. Ze concluderen dat er in die tijd wel al benchmarking tools waren voor high-end devices maar not niet voor low-end zoals mobile devices. Tegenwoordig zijn er al 3D becnhmarking tools die voorzien zijn op low-end architectures zoals mobile devices. Eén van deze benchmarking tools is 3D Mark deze word ook aanzien als één van de beter benchmarking tools voor 3D.  