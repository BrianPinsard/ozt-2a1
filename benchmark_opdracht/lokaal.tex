In dit deel zullen de resultaten van de testen toegelicht worden, deze zijn onderverdeeld in: de grafische testen, de Physics test en de algemene scores berekend volgens de formules vermeld in sectie 3.

\subsection{Grafische testen}

Deze test legt de nadruk eerder op de performantie van de grafische kaart, als we de laptops bekijken dan mogen we vrij zeker zijn dat de MSI GT72 er bovenuit zal steken door zijn nieuwere en sterke grafische kaart. Hier ziet u de bijhorende grafiek: \\
\includegraphics[width=8cm]{grafische}
\vskip 0.1cm
\noindent
Zoals verwacht steekt de MSI GT72 2QE hier met kop en schouders bovenuit door zijn grafische kaart (GTX980M) deze is voor het moment de nieuwste versie voor de NVIDIA chipset.
Alhoewel de MSI GE60 en de Acer Aspire V3 772G dezelfde grafische kaart hebben is er een klein verschil in performantie, de MSI GE60 zijn performantie ligt iets hoger.\\
Het vreemde is dat de Acer Aspire E17 een nieuwere grafische kaart heeft maar toch een gelijkaardige performantie heeft tegenover de andere twee.\\ Dit komt doordat de GPU's die eindigen met 50 (en hoger) behoren tot de mid- tot high-end klasse maar die eronder worden verondersteld van low-end te zijn.

\subsection{Physics test}
Zoals vermeld in sectie~\ref{sec-moa} is deze test meer op de Processor gericht, als we naar de tabel kijken zien we dat de MSI GT72 2QE de nieuwste generatie CPU heeft, de MSI GE60 en de Acer Aspire V3 hebben dezelfde generatie CPU en de Acer Aspire E17 heeft een iets zwakkere CPU.\\
\includegraphics[width=8cm]{physics}
\vskip 0.1cm
\noindent
Deze test wordt opgedeeld in vier testen zoals vermeld in sectie~\ref{sec-moa-phys}, we zien dat de MSI GT72 in het eerste level een veel hogere performantie heeft dan de rest, verassend zien we ook dat de MSI GE60 een aanzienlijk lagere score heeft als het om weinig objecten tegelijk draait.\\
Wanneer we het tweede level bekijken zien we dat de performantie van de Acer Aspire E17 heel hard is afgevallen en dat de MSI GE60 en Acer Aspire V3 min of meer gelijkgekomen zijn maar de Acer Aspire V3 steekt er nog steeds iets boven.\\
Bij het derde niveau is de MSI GT72 zijn performantie sterk aan het gelijk komen met die van de MSI GE60 en de Acer Aspire V3 terwijl de Acer Aspire E17 verder afvalt.\\
Bij het laatste niveau kunnen we concluderen dat deze CPU's weinig onderscheid hebben als het draait om zoveel berekeningen tegelijk. Er is geen resultaat bij level 4 bij de Acer Aspire E17 omdat het aantal frames per seconde dat de laptop haalt onder de minimum threshold valt voor deze test. Dit zou betekenen dat het ongezond zou zijn voor de processor om deze test te vervolledigen. \citep{3dmark_tech}

\subsection{Scores}
Op het einde  van de testen maakt Skydiver een score voor de grafische testen, de Physics test, voor de gecombineerde test en daarna wordt ook een algemene score opgesteld. Dit volgens de formules die beschreven zijn door \cite{3dmark_tech}. \\
\includegraphics[width=8cm]{algemeen}
\vskip 0.1cm
\noindent
Bij de Grafische testen score zien we dat de MSI GT72 er bovenuit steekt, dat de MSI GE60 en de Acer Aspire V3 ongeveer gelijk zitten en dat de Acer Aspire E17 iets onder de scores van de MSI GE60 en de Acer Aspire V3 ligt.\\
Bij de Physics score zien we dat de MSI GT72, MSI GE60 en de Acer Aspire V3 een gelijkaardige score hebben terwijl de Acer Aspire E17 de helft van hun scores heeft.\\
Bij de gecombineerde score zien we dan dat de MSI GE60 en de twee Acer Aspires samen liggen en de MSI GT72 de score van hun verdrievoudigt.\\
De algemene score is na de andere scores te zien ook al vrij duidelijk, de MSI GT72 ligt bovenaan en de andere drie liggen ongeveer op hetzelfde niveau van elkaar.